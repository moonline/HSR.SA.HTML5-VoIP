\section{Persönlicher Schlussbericht Tobias Blaser}	
	Persönlich habe ich sehr viel gelernt, vor allem im Bereich JavaScript Architektur und Frameworks. Auch im Bereich CSS habe ich mir einiges an neuem Wissen aneignen können, vor allem was Flexbox und LESS betrifft.
	
	Das Erforschen einer sehr modernen Technologie zog die Vorteile mit sich, das als Zielgruppe nur moderne Browser zum Einatz kommen konnten. Dies hat sich sehr positiv auf die Entwicklung ausgewirkt. Wir konnten modernste CSS Funktionalität und JavaScript Schnittstellen verwenden und mussten uns keine Sorgen um nicht unterstützende Browser machen, was sehr angenehm war.
	
	\subsection{Teamarbeit}		
		Unangenehm war die Unsicherheit zu Begin der Studienarbeit wegen der entzogenen Projektzulassung meines ursprünglichen Partners.
	
		Trotz dieser anfänglichen Unsicherheiten hat die Zusammenarbeit nach dem Teamwechsel sehr gut funktioniert. Wir haben beide sehr engagiert gearbeitet und viel Zeit investiert. Besonders in den letzten Wochen.
		
		Die Teamkoordination war angfangs allerdings nicht immer einfach, da wir an unterschiedlichen Tagen an der SA arbeiteten. Nachdem ich meine Büro-Arbeitstage geschoben hatte, wurde das Zusammenarbeiten wesentlich einfacher.

		Auch die Zusammenarbeit mit Herrn Bläser war sehr angenehm. Wir haben sehr viel Unterstützung und wertvolle Inputs erhalten und Herr Bläser hat ein Code Review der Applikation durchgeführt. Hierfür möchte ich mich an dieser Stelle herzlich bedanken.
			
	
	\subsection{Fazit}		
		Für die knappe Zeit hatten wir uns sehr viel vorgenommen. Trotzdem haben wir es geschafft, sogar noch optionale Features einzubauen.
		
		Das Projekt hat eine Menge Spass gemacht, auch wenn es zwischendurch Moment gab, in denen mich JavaScript durch seine Eigenheiten einiges an Nerven gekostet hat.
	
		Nicht ganz so viel Spass hat das Finishing der Dokumentation gemacht. Vor allem weil durch Abstract, Management Summary, Hauptteil und Anhang viele Themen vier Mal durchgekaut werden mussten um sie in unterschiedlicher Kompaktheit zu produzieren.

		
	
	