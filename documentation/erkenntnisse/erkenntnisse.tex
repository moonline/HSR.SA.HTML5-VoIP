\chapter{Erkenntnisse}
	Die während der Entwicklung der Architekturprototypen und dem Ausbau der Connection gesammelten Erkenntnisse sind hier zusammengetragen:

	\section{Media Streaming}
		\begin{description}
			\item[Verbindungsabbruch] Bricht die Verbindung ab, so bleibt die Video- und Audiowiedergabe stehen, bis die Verbindung wieder da ist und die UDP Pakete wieder ankommen.
		
			\item[Downsampling] Der Mediastream wird automatisch mit Minimalqualität gestartet und abhängig
	 		von Prozessorleistung und Bandbreite hochgefahren, bis zu einer Datenrate von etwa 300MiB/s.
	 		
	 		\item[Resolution Properties] Obwohl die Spezifikation für getUserMedia eine Möglichkeit zur Festlegung der gewünschten Auflösung definiert, ist dies in den Browsern bisher nicht implementiert.
	 		
	 		\item[Audio/Video attach/detach] Hinzufügen und Entfernen von Streams sollte eigentlich möglich sein. Die
	 		Browser implementieren dies bisher jedoch nicht. Gemäss dem Mozilla-Blog soll dieses Feature jedoch demnächst umgesetzt werden.
	 		
	 		\item[Autorotation Video] Wird als Endgerät ein Smartphone oder Tablet mit integriertem Beschleunigungssensor genutzt, so wird das Video automatisch gedreht, sobald das Smartphone gedreht wird. Dadurch ist der Teilnehmer immer in korrekter Ansicht zu sehen, egal ob das Device Quer- oder Hochformat gehalten wird.
	 		Auch das Videoelement beim Empfänger erhält die Information über die Änderung des Videoformates und skaliert das Videoframe entsprechend, sofern die Grösse nicht durch Styles übersteuert wurde.
	 	\end{description}
	 	
	 \section{Peerconnection}
	 	Eine Peer-Connection kann nur durch den Austausch von SDP aufgebaut werden. Innerhalb der SDP werden Parameter für die Peer-Connection
	 	sowie IP und allenfalls Endpunktkandidaten übertragen.
	 	
	 	\begin{description}
			\item[Endpunktkandidaten] Je nach Implementation übertragen die Browser die Endpunktkandidaten (Candidates) einzeln und nicht in einer bestimmten Reihenfolge. Ist die Verbindung noch nicht initialisiert worden, so müssen diese zwischengespeichert werden, bis die Verbindung bereit ist.

			\item[Verschlüsselung] Der übertragene Stream ist standardmässig verschlüsselt. Die Verschlüsselung ist Teil des Standards und kann nicht abgeschaltet werden.
			
			\item[Verbindungsaufbau] Ice Candidates müssen einer RTCPeerConnection attatched werden, bevor eine Offer oder eine Answer generiert wird. Ansonsten hat der andere Endpunkt zu wenig Informationen um eine Verbindung aufzubauen. Die Ice Candidates sind Proposals für Verbindungsendpunkte (Ports). Die Browser senden einander mehrere möglich Kombinationen für jede Art von Verbindung (DataChannel, Video, Audio). Anschliessend werden alle Kombinationen probiert, bis eine funktioniert.
	 	\end{description}
	 
	 \section{Filezugriff}
	 	Dateizugriff ist in JavaScript nur über einen User-Event möglich (Upload
	 	Field oder Drag/Drop). Daher ist es nicht möglich, Adressbücher auf die
	 	Festplatte abzulegen und wieder zu öffnen, ohne den User dazu aufzufordern.
	 	
	 
	 \section{Offline-Einschränkungen}
	 	\begin{description}
			\item[STUN-Services] Ohne eigener STUN-Service ist eine Internetverbindung
			zwingend. Für Telefonie in einem abgeschotteten Netz wird ein eigener
			STUN-Service benötigt. Es ist nicht möglich auf die STUN-Services zu
			verzichten und die Adresse manuel zu setzen, weil das STUN-Protokoll auch den
			P2P-Verbindungsaufbau übernimmt.
	 	\end{description}
	 		
	 \section{SIP}
	 	\begin{description}
			\item[WebSockets] Obwohl der OpenSource-SIP-Server Kamailio WebSocket als Feature aufführt,
	 		ist die Implementation noch nicht vollständig, sodass SIP-Kommunikation über
	 		WebSockets noch unmöglich ist.
	 	\end{description}
	 		
	\section{Browserunterschiede}
		\begin{description}
			\item[ICE Candidates] Firefox packt die ICE Candidates\footnote{Vorschläge
			für Ports} direkt in die Session Description ein. Chrome schickt sie einzeln.
			Da Chrome nicht auf die Reihenfolge achtet, kann es passieren, dass ICE
			Candidates vor einer Offer ankommen. Aus diesem Grund müssen ankommende ICE
			Candidates in einer Queue gebuffert werden, bis eine Offer ankommt.
		
			\item[DataChannel] Firefox unterstützt DataChannel\footnote{P2P-Datenkanal parallel zum Video-
			und Audiostream} bereits vollständig. Chrome unterstützt ihn nur im Modus
			"`reliable: false"'. Setzt man ihn auf "`reliable: true"', kommen die Events
			nicht zuverlässig an.
			
			\item[Media Access] Firefox erlaubt mehrfachen Kamera-Access und teilt sich
			die Kamera auch problemlos mit anderen Prozessen. So ist es im Firefox
			möglich, von einem Tab in einen anderen anzurufen. Chrome will die Kamera komplett für sich beanspruchen und wirft eine Fehlermeldung, falls sie bereits von einem anderen Prozess genutzt wird.
			
			\item[Schnittstellenbezeichnung] Die Schnittstellen unterstützen beide
			Browser erst mit Präfix. Die offizielle Bezeichnung wird noch nicht unterstützt. Adapter.js wurde dafür entwickelt und mappt die offiziellen Schnittstellenbezeichnungen auf die mit Prefix, sodass in der Applikation trotzdem mit den offiziellen gearbeitet werden kann und auch Kompatibilität besteht, sobald die Browser auf die definitiven Schnittstellen umstellen.
	 	\end{description}