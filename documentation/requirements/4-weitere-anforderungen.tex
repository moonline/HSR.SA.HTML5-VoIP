\chapter{Weitere Anforderungen}

\section{Qualitätsmerkmale}
Die Applikation soll intuitiv bedienbar sein und die Video- und Audio Auflösung soll abhängig von der Verbindungsqualität skaliert werden.


\subsection{Functionality}
\begin{itemize}
	%\item[Korrektheit:] 
	%\item[Angemessenheit:] 
	\item[Interoperabilität:] Die Applikation soll auf existierende Standards setzen und wo eigene Standards nötig sind, diese offen dokumentieren.
	\item[Sicherheit:] Die Audio- und Videokommunikation sowie die Verbindung zum Server soll nach Möglichkeit verschlüsselt erfolgen.
\end{itemize}

%\subsection{Efficiency}
%\begin{itemize}
%	\item[Wirtschaftlichkeit:] 
%	\item[Zeitverhalten:] 
%\end{itemize}

\subsection{Usability}
\begin{itemize}
	\item[Verständlichkeit:] Die Benutzeroberfläche soll selbsterklärend aufgebaut sein. Knöpfe und Labels sollen sprechend bezeichnet sein.
	%\item[Bedienbarkeit:] 
	\item[Robustheit:] Die Oberfläche soll nicht hängen bleiben bei länger dauernden Operationen und auf Fehler angemessen reagieren.
\end{itemize}

%\subsection{Reliability}
%\begin{itemize}
%	\item[Reife:] RC 1
%	\item[Fehlertoleranz:] 
%	\item[Wiederherstellbarkeit:] 
%\end{itemize}

\subsection{Portability}
\begin{itemize}
	\item[Anpassbarkeit:] Die Applikation soll so gestaltet werden, das sich die Unterstützung für weitere Technologien einfach hinzufügen lässt.
	\item[Installierbarkeit:] Die Applikation soll auf jedem WebRTC fähigen modernen Browser laufen, der sich an die Standard Darfts hält.
	\item[Austauschbarkeit:] Die Oberfläche soll durch eigene Styles einfach umgestaltet und angepasst werden können.
\end{itemize}

%\subsection{Maintainability}
%\begin{itemize}
%	\item[Analysierbarkeit:] 
%\end{itemize}

\section{Schnittstellen}

\subsection{Signalling Schnittstelle}
Die Applikation soll als Signalling Technologie einen offenen Standard (z.B. SIP oder XAMPP) verwenden.

\subsection{Telefonbuchschnittstelle}
Die Telefonbuchschnittstelle soll offene und verbreitete Standardformate unterstützen (z.B. vcard, csv, json, ...).
