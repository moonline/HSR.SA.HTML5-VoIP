\chapter{Performanceanalyse}
	\section{Leistungsverbrauch}
		Zur Analyse der Performance wurden Versuche mit Desktoprechnern und Mobilgeräten gemacht. Erwartet wurde, das die Performance auf den Desktopgeräten wesentlich besser ist als bei den Mobilgeräten.
		Angeschaut wurden die CPU-Belastung, RAM-Verbrauch und Netzwerktrafic sowie Audio- und Videoqualität.
		
		Es hat sich gezeigt, das die Performance sehr stark von der Leistung des Prozessors abhängt. Die Vermutung, das Desktopgeräte besser performen als Mobilgeräte war jedoch falsch. 
		Auf den Desktoprechnern verbraucht ein P2P Stream ca. 10-20\% CPU Leistung, auf dem Netbook und dem Tablet wesentlich mehr. Das verwendete Smartphone lag näher bei den Desktopgeräten, wurde allerdings spürbar warm.	
		
	\section{Speicherverbauch und Netzwerktrafic}
		Der Speicherverbauch ist bei allen Geräten vernachlässigbar und im Systemmonitor praktisch nicht zu sehen.
		
		Der Netzwerktrafic hängt von zwei Faktoren ab: Wie leistungsfähig der Prozessor ist und die verfügbare Bandbreite im Netz. 	
		Videodecodierung und Verschlüsselung verbrauchen in den aktuellen Versionen von Firefox und Chrome noch viel Leistung. Netbook und Tablet hatten entsprechend Mühe, einen Datenstrom mit hoher Datenrate zu liefern. Das Empfangen machte keinem der Geräte Probleme.
	
	\section{Zusammenfassung}
		Die folgende Tabelle fasst die Bedingungen kurz zusammen, unter denen gute und akzeptable Performance möglich sind.
		
		\noindent
		\begin{tabularx}{\linewidth}{|X|X|X|X|}
			\hline
			& \textbf{Gute Performance} & \textbf{Akzeptable Performance} & \textbf{Schlechte Performance} \\
			\hline
			\textbf{Audio/Video} & • Video flüssig\newline• Audio verständlich & • Video Einzelbilder\newline• Audio verständlich & • Nur Audio möglich\\
			\hline
			\textbf{Bandbreite zwischen den Peers} &> 50KiB/s &> 25Kib/s &< 25KiB/s \\
			\hline
			\textbf{Prozessor} &> 1.5Ghz, 2 Core &> 1Ghz, 2 Core &weniger Leistung \\
			\hline
			\textbf{Kamera Auflösung} &> 640 x 480 Pixel (Maximale Auflösung in den aktuellen Implementationen) & & \\
			\hline
		\end{tabularx}
		
		\vspace{0.5cm}
		Bei schwächeren Geräten und einer kleineren möglichen Datenrate beeinträchtigt das Video die Audioübertragung zu stark. Entsprechend lohnt es sich, unter diesem Umständen das Video gar nicht zu aktivieren.
			
		Ausführliche Performanceanalyse siehe Anhang \ref{performanceanalyse}
	
	\section{Fazit}
		Videocodierung und Verschlüsselung benötigen noch sichtlich viel Rechenleistung. Desktopgeräte und Mobilgeräte im mittleren- und oberen Leistungssegment sind allerdings in der Lage, die notwendige Leistung zu liefern.
		
		Die bevorzugten Codecs, die auf der VP8-Engine basieren, werden bisher von keinem Gerät hardwareseitig unterstützt. Nach der Veröffentlichung einer OpenSource-Implementation des Codecs h.264
		durch Cisco ist zu erwarten, das dieser Einzug in die Browser hält und durch die bereits existierenden Hardwarechips die erforderliche Rechenleistung enorm senken wird.
			
		Ebenfalls ist zu erwarten, das zukünftige Implementation der Videocodecs ohne Hardwareunterstützung einen Teil über die GPU rendern und damit erhebliche Performanceverbesserungen bringen werden.
			
	
	
\chapter{Firewall Testing}
	\section{Signaling}
		Je nach verwendetem Protokoll wird dieses in einer stark reglementierten
		Umgebung geblockt. Ebenfalls kann es vorkommen, das beispielsweise
		WebSockets-Pakete unterwegs verworfen werden, weil Router sie aus
		Policygründen nicht weiterleiten.
		
		Wird für das Signaling ein HTTP-basierter Mechanismus verwendet, sollten
		keine Pakete geblockt werden.

	\section{P2P Connection}
		Für den Aufbau der Peer-to-Peer Connection ermitteln beide Clients über einen
		STUN-Service ihre externe IP, tauschen diese über den Signalingchannel
		mittels SDP\footnote{Session Description Protocol} aus und verbinden sich
		direkt P2P um Parameter für die Verschlüsselung auszutauschen. Für diesen
		Austausch wird ebenfalls das UDP basierte STUN-Protokoll verwendet.
		In starkt reglementierten Umgebungen werden unter Umständen diese STUN-Pakete
		geblockt. Dadurch kann keine Verbindung zwischen den Clients aufgebaut werden.
		
		In gewöhnlichen Umgebungen sollten STUN-Pakete jedoch nicht geblockt werden.
	
	\section{Firewall Pass Tests}
		Siehe Anhang \ref{firewalltests}