\chapter{Risikomanagement}
\section{Risiken}

Im Folgenden werden die zu erwartenden Risiken aufgeführt und Massnahmen im Falle deren Eintretens gezeigt. Die akkumulierte Reservezeit aller gewichteten Risikoschäden beläuft sich auf 
24 Stunden.

\begin{table}[h!]
	\centering
	%Die Spalten werden aufgeteilt auf 0.3 resp. 0.7-fache Textbreite
	\begin{tabular}{|p{0.3\textwidth} | p{0.7\textwidth} |}
	\hline	
	Risiko-ID & 1 \\
	\hline
	Titel & Implementierungshindernis \\
	Beschreibung & Aufgrund einer nicht bedachten Schwierigkeit, verzögert sich die Entwicklung des Programms \\
	max. Schaden	& 10h \\
	Eintrittswahrscheinlichkeit & 0.4 \\
	Gewichteter Schaden	& 4h \\
	Vorbeugung	& Sorgfältige Abklärungen im Vorfeld der Implementierung \\
	Massnahmen	& Zusätzliche Entwicklungszeit, um das Problem zu lösen \\
	\hline
	\end{tabular}
\end{table}

\begin{table}[h!]
	\centering
	\begin{tabular}{|p{0.3\textwidth} | p{0.7\textwidth} |}
	\hline
	Risiko-ID & 2 \\
	\hline
	Titel & Fehleinschätzung \\
	Beschreibung & Der Zeitaufwand für einzelne Arbeitspakete wurde falsch eingeschätzt \\
	max. Schaden	& 15h \\
	Eintrittswahrscheinlichkeit & 0.6 \\
	Gewichteter Schaden	& 9h \\
	Vorbeugung	& Aufwand für die Pakete wird im Team geschätzt um einen optimalen Schätzwert zu erhalten \\
	Massnahmen	& Kritische Arbeitspakete werden zuerst bearbeitet, optionale Features dann nach Möglichkeit implementiert \\
	\hline	
	\end{tabular}
\end{table}

\begin{table}[h!]
	\centering
	\begin{tabular}{|p{0.3\textwidth} | p{0.7\textwidth} |}
	\hline
	Risiko-ID & 3 \\
	\hline
	Titel & Performance \\
	Beschreibung & Das Abrufen der Website kann nicht in nützlicher Frist durchgeführt werden\\
	max. Schaden	& 50h \\
	Eintrittswahrscheinlichkeit & 0.2 \\
	Gewichteter Schaden	& 10h\\
	Vorbeugung	& Es wird von Anfang an möglichst ressourcenschonend programmiert \\
	Massnahmen	& Optimierung der Algorithmen\\
	\hline	
	\end{tabular}
\end{table}

\begin{table}[h!]
	\centering
	\begin{tabular}{|p{0.3\textwidth} | p{0.7\textwidth} |}
	\hline
	Risiko-ID & 4 \\
	\hline
	Titel & Webserver blockieren Serienabfragen \\
	Beschreibung & Es kann sein, das einzelne Webserver unseren Client blockieren wenn zu viele Anfragen in zu kurzer Zeit erfolgen \\
	max. Schaden	& 10h \\
	Eintrittswahrscheinlichkeit & 0.1 \\
	Gewichteter Schaden	& 1h \\
	Vorbeugung	& Alternativszenario mit zeitverzögerten Abfragen wird in Betracht gezogen \\
	Massnahmen	& Obiges Szenario wird anstelle der schnellstmöglichen Abfragegeschwindigkeit implementiert \\
	\hline	
	\end{tabular}
\end{table}

\begin{table}[h!]
	\centering
	\begin{tabular}{|p{0.3\textwidth} | p{0.7\textwidth} |}
	\hline
	Risiko-ID & 5 \\
	\hline
	Titel & Pluginarchitektur \\
	Beschreibung & Ob und wie eine Pluginarchitektur unter Java funktioniert ist unbekannt. \\
	max. Schaden	& Projektfehlschlag 120h \\
	Eintrittswahrscheinlichkeit & 0.1 \\
	Gewichteter Schaden	& 12h \\
	Vorbeugung	& Funktion einer Pluginarchitektur unter Java frühzeitig abklären. \\
	Massnahmen	& Grundsätzliche Funktion vor weiterer Planung abklären. \\
	\hline	
	\end{tabular}
\end{table}

\clearpage	

\section{Umgang mit Risiken}
Das Projektteam wurde über sämtliche oben erfassten Risiken informiert und ist sich über die zu ergreifenden Massnahmen im Schadenfall im Klaren.
Weitere, in der Evaluierung übersehene Risiken und deren Schäden, werden im Falle deren Auftretens vom Gesamtteam behandelt und ausgebessert.
