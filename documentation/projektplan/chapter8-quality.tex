\chapter{Qualitätsmassnahmen}

\section{Dokumentation}
Die Dokumentation wird in \LaTeX\ verfasst. Dies bietet wesentliche Vorteile:
\begin{itemize}
	\setlength{\itemsep}{-\parsep}
	\item Plaintextsourcefiles, die mit SVN versioniert und bei unterschiedlichen Zuständen problemlos zusammengeführt werden können
	\item Wenig Speicherplatzverbrauch durch die Sourcefiles
	\item Durch den Verzicht auf eine Office Umgebung ist die Wahrscheinlichkeit eines Absturzes und eine Beschädigung der Dateien geringer
	\item Das Projektteam muss keine unnötige Zeit für Layouting mit einem Office Programm und die damit verbundenen Probleme verwenden oder ein Desktop Publishing Programm einsetzen. Dank einer ausgereiften Vorlage wird der Layoutingaufwand mit \LaTeX\ klein gehalten.
	\item Das Projektteam kann die Dokumentation aufteilen und die einzelnen Teile in die Hauptdatei einbinden.
\end{itemize}
Die Qualität wird dadurch sichergestellt, dass jedes Projektmitglied die Qualität für die von ihm produzierten Teilstücke gewährleistet und gleichzeitig die Dokumentation als Ganzes vom gesamten Team überprüft wird. Der Projektleiter gewährleistet, dass keine Teilbereiche bei der Abgabe fehlen.

\section{Projektmanagement}
Zur Unterstützung des Projektmanagements wird Redmine mit SVN Anbindung eingesetzt. Verwaltet werden sollen Arbeitspakete, Meilensteine, Einteilung von Teammitgliedern, Projektstand und Projektenwicklung.

\subsection{Gastzugang Redmine}
\label{text:redmine}
Zugang: http://152.96.56.72/redmine/\\
Benutzer: supervisor\\
Passwort: supervisor \\

Der Supervisor hat die Rechte um alle Beiträge, Tickets, Dokumente, sowie den Kalender und die Gantt-Chart einzusehen und nötigenfalls zu kommentieren, jedoch keinerlei Schreib- oder Verwaltungsrechte.

\section{Entwicklung}
Der Programmcode wird mit SVN versioniert. Die Qualität soll einerseits durch den entsprechenden Entwickler selbst, andererseits durch Reviews, Struktur Checks und Funktionstests gewährleistet werden.

\subsection{Vorgehen}
Bei der Entwicklung wird nach dem RUP vorgegangen. Insbesondere liegt das Augenmerk auf einer iterativen und agilen Entwicklung. Für bekannte Probleme sollen standardisierte Software Patterns angewendet werden. Auch die Prinzipien der GoF sollen in die Entwicklung einfliessen.

\subsection{Unit Testing}
Das Zusammenwirken der Klassen, insbesondere die Interaktion mit dem Repositoryserver und die Kommunikation mit einer beliebigen Webseite sollen durch Unit Tests überprüft werden. Zudem soll jeder Check verschiedene Unit Tests bestehen müssen.

\subsection{Code Reviews}
Jeder Programmteil soll Reviews unterzogen werden. Das Hauptaugenmerk bei den Reviews liegt auf Pre-/Postconditions, unerlaubten und unerwarteten Werten, Seiteneffekten sowie dem Einhalten der Code Style Guidelines.
Die Reviews werden ohne den betreffenden Programmierer stattfinden, um zu verhindern, dass ein "Vorstellen / Erklären"\ des Codes durch den Entwickler stattfindet. Programmcode und zugehörige Kommentare müssen selbsterklärend aufgebaut sein.

\subsection{Code Style Guidelines}
Für die Entwicklung sollen die Oracle Code Style Conventions angewendet werden: \\ \href{http://www.oracle.com/technetwork/java/codeconv-138413.html}{www.oracle.com/technetwork/java/codeconv-138413.html}

\section{Testen}

\subsection{Testumgebung}
Als Testumgebung soll von Hand ein HTML-Seiten Gerüst aufgebaut werden. Jede Seite enthält eine Anzahl verschiedener Fehler. Getestet wird, ob und wie gut die Applikation die Fehler erkennt. Die Fehler sollen von verschiedenen Schwierigkeitsgraden sein.

\subsection{Unit Tests}
Die Kernfunktionalitäten sollen mit Unit Tests überprüft werden. Getestet wird mit JUnit 4. Gearbeitet wird mit je drei Tests pro Testszenario.

\subsubsection{Was soll getestet werden}
Klassen, die Funktionen besitzen, die über Setter und Getter hinausgehen müssen getestet werden. Es sollen alle Methoden getestet werden, die mehr als fünf Zeilen lang sind und funktional mehr als ein Setter oder ein Getter ausführen. 

\subsubsection{Wie soll getestet werden}
Die Funktion der entsprechenden Methoden soll mit einem Normalfall, Fehlerfall und einem oder mehreren Randfällen getestet werden.

\subsection{Pluginchecks}
Jedes Plugin muss einige Tests bestehen, damit das korrekte Zusammenarbeiten mit der Applikation gewährleistet werden kann. Getestet werden soll, ob das Plugin korrekt angesprochen werden kann und ein korrektes Resultat in der vorgeschriebenen Form liefert. Zusätzlich soll mit drei Testszenarien, den Fähigkeiten des Plugins entsprechend, die Funktionalität überprüft werden. Die Szenarien testen den Normalfall, den Fehlerfall und einen oder mehrere Randfälle.

\subsection{Usability Test}
Die Benutzbarkeit der grafischen Oberfläche soll mittels Usability Test mit drei Testpersonen überprüft werden.

\subsection{WorstCase Test}
Getestet werden sollen Worst Case Szenarien, unabhängig der Klasse: 
\begin{itemize}
	\setlength{\itemsep}{-\parsep}
	\item Keine Internetverbindung vorhanden
	\item Angegebenes Repository fehlerhaft oder nicht vorhanden
	\item Plugin defekt
	\item Fehlen der Settings-Datei
	\item Zu checkende Seite, die grösser als der Arbeitsspeicher ist
	\item Sehr langsame Internetverbindung
\end{itemize}
Keiner der angegebenen Fälle darf das Programm zum Absturz bringen. Ausserdem sollen die Fehler in jeder Situation zuverlässig geloggt werden.
Pro Konfiguration sollen drei Tests durchgeführt werden.

\subsection{Systemtest}
Das System soll mit zehn verschiedenen, realitätsnahen Szenarien als ganzen getestet werden (Blackboxtest). 
