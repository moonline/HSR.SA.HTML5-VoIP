\chapter{Projekt Übersicht}
Mit Hilfe des "<Software-Engineering 2 - Projektes"> wollen wir eine Applikation mit Namen "<PhipS"> erstellen. "<Phips"> ermöglicht es Webentwicklern, Fehler auf Webseiten aufzufinden. Dabei wird jede einzelne Seite einer Webseite abgearbeitet und verlinkte Dateien auf Verfügbarkeit geprüft. Ebenso wird die Webseite auf fehlende Teiltemplates geprüft. Fehler dieser Art werden oftmals nicht entdeckt, da ein \ac{CMS}, einen korrekten Header und den Rest der Seite ausgibt, auch wenn ein Teil fehlt.

Die Benutzerfreundlichkeit von "<Phips"> werden wir insofern garantieren, indem wir eine ansprechende und übersichtliche grafische Benutzeroberfläche (kurz \ac{GUI}) gestalten. Zusätzlich werden wir den Bericht nach der Webseitenprüfung in verschiedenen ebenfalls übersichtlichen Formaten zu Verfügung stellen.

"<Phips"> selbst besteht aus einer Grundapplikation und einzelnen Zusatzplugins. \\
Der Grundgedanke dabei ist, dass der Benutzer in den Einstellungen auswählen kann, welche Plugins er herunterladen und installieren will. Dabei wird eine Verbindung mit einem Repository-Server aufgebaut, wo sich die Plugins befinden. Der Benutzer wird also sobald er die Einstellungen bearbeitet automatisch darauf hingewiesen, ob weitere Plugins verfügbar sind oder ob sogar eine verbesserte und aktuellere Version verfügbar ist. Es besteht jedoch auch die Möglichkeit, ein lokales Verzeichnis anzugeben, um selbst erstellte Plugins einzubinden.

Damit unsere Applikation verhältnismässig schnell Resultate liefert, werden die Plugins, welche die einzelnen Seiten überprüfen, parallel ausgeführt.
\section{Zweck und Ziel}
\begin{itemize}
 \item Eine effiziente und einfache Möglichkeit für Webentwickler und Contentmanager bieten, um Webseiten auf inhaltliche Fehler zu untersuchen.
 \item Vorteile von Unit-Testing in den Bereich der Webentwicklung bringen, um die Qualität und Zuverlässigkeit des Contents von Webseiten verbessern.
 \item Wissen aus den Modulen "<Software-Engineering 1"> und "<Parallel- und Netzwerkprogrammierung"> in die Praxis umsetzen, um Erfahrungen im Bereich der Softwareentwicklung zu sammeln.
 \item Programmierkenntnisse und Zusammenarbeit in einem Team fördern.
\end{itemize}
\section{Lieferumfang}
Unsere Software umfasst grundlegende Tests wie die Prüfung auf fehlerhafte Links, fehlende Bild-Ressourcen, fehlende Teiltemplates oder inkorrekte Email-Adressen.

\subsection{mögliche Erweiterungen}
\begin{itemize}
	\setlength{\itemsep}{-\parsep}
 \item Report Export Funktion in \ac{HTML}, \ac{CSV} oder \ac{JSON}
 \item Settings (z.B. Maximale Anzahl prüfende Seiten, Maximale Rekursionstiefe, Maximale Laufzeit, ...)
 \item \ac{i18n}
 \item Automatisierte Plugin Updates
 \item Metrikprüfung
 \item Statistik
\end{itemize}
\section{Annahmen und Einschränkungen}
Pro Teammitglied rechnen wir mit einem Zeitaufwand von 120 Stunden. Sollten wir das Projekt in weniger Stunden fertiggestellt haben, so werden optionale Erweiterungen realisiert beziehungsweise bei Knappheit der Zeit weggelassen.

Der späteste Abgabetermin ist am Freitag, 1. Juni 2012, um 17.00 Uhr.
