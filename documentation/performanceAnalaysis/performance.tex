\chapter{Performanceanalyse}
	\section{Leistungsverbrauch}
		Zur Analyse der Performance wurden Versuche mit Desktoprechnern und Mobilgeräten gemacht. Erwartet wurde, dass die Performance auf den Desktopgeräten wesentlich besser ist als bei den Mobilgeräten.
		Angeschaut wurden die CPU-Belastung, RAM-Verbrauch und Netzwerktrafic sowie Audio- und Videoqualität.
		
		Es hat sich gezeigt, das die Performance sehr stark von der Leistung des Prozessors abhängt. Die Vermutung, dass Desktopgeräte besser performen als Mobilgeräte war jedoch falsch. 
		Auf den Desktoprechnern verbraucht ein P2P Stream ca. 10-20\% CPU Leistung, auf dem Netbook und dem Tablet wesentlich mehr. Das verwendete Smartphone lag näher bei den Desktopgeräten, wurde allerdings spürbar warm.	
		
		
	\section{Speicherverbauch und Netzwerktrafic}
		Der Speicherverbauch ist bei allen Geräten vernachlässigbar und im Systemmonitor praktisch nicht zu sehen.
		
		Der Netzwerktrafic hängt von zwei Faktoren ab: Leistungsfähigkeit des Prozessors und effektiv verfügbare Bandbreite zwischen den Peers.
		Videodecodierung und Verschlüsselung verbrauchen in den aktuellen Versionen von Firefox und Chrome noch viel Leistung. Netbook und Tablet hatten entsprechend Mühe, einen Datenstrom mit hoher Datenrate zu liefern. Das Empfangen bereitete keinem der Geräte Probleme.
		
		Die Browser fahren die Videoauflösung langsam hoch, sobald sie feststellen, das genügend Leistung und Bandbreite vorhanden ist. Dies lässt sich gut zeigen, indem ein verbundenes Smartphone gedreht wird. Dabei wird das Video ebenfalls gedreht und neu dargestellt, wobei die Auflösung und der Bandbreitenverbrauch ab einem Minimum von ca. 50KiB/s kontinuierlich ansteigen und sich bei ca. 300 KiB/s einpendeln.
	
	
	\section{Zusammenfassung}
		Die folgende Tabelle fasst die Bedingungen kurz zusammen, unter denen gute und akzeptable Performance möglich sind:

		\noindent
		\begin{tabularx}{\linewidth}{|X|X|X|X|}
			\hline
			& \textbf{Gute Performance} & \textbf{Akzeptable Performance} & \textbf{Schlechte Performance} \\
			\hline
			\textbf{Audio/Video} & • Video flüssig\newline• Audio verständlich & • Video Einzelbilder\newline• Audio verständlich & • Nur Audio möglich\\
			\hline
			\textbf{Bandbreite zwischen den Peers} &> 50KiB/s &> 25Kib/s &< 25KiB/s \\
			\hline
			\textbf{Prozessor} &> 1.5Ghz, 2 Core &> 1Ghz, 2 Core &weniger Leistung \\
			\hline
			\textbf{Kamera Auflösung} &> 640 x 480 Pixel (Maximale Auflösung in den aktuellen Implementationen) & & \\
			\hline
		\end{tabularx}
		
		\vspace{0.5cm}
		Bei schwächeren Geräten und einer kleineren möglichen Datenrate beeinträchtigt das Video die Audioübertragung zu stark. Entsprechend lohnt es sich, unter diesem Umständen das Video gar nicht zu aktivieren.
		
		Während dem Entwickeln ist aufgefallen, das die Performance von Chrome etwas besser ist als die von Firefox. Ist die Performance auf einem Gerät mit Firefox zu schlecht, so kann man es mit Chrome erneut versuchen.
			
		Ausführliche Performanceanalyse siehe Anhang \ref{performanceanalyse}.
	
	
	\section{Fazit}
		Videocodierung und Verschlüsselung benötigen noch sichtlich viel Rechenleistung. Desktopgeräte und Mobilgeräte im mittleren- und oberen Leistungssegment sind allerdings in der Lage, die notwendige Leistung zu liefern.
		
		Die bevorzugten Codecs, die auf der VP8-Engine basieren, werden bisher von keinem Gerät hardwareseitig unterstützt. Nach der Veröffentlichung einer OpenSource-Implementation des Codecs h.264 durch Cisco ist zu erwarten, das dieser Einzug in die Browser hält und durch die bereits existierenden Hardwarechips die erforderliche Rechenleistung enorm senken wird.
			
		Ebenfalls ist zu erwarten, das zukünftige Implementation der Videocodecs ohne Hardwareunterstützung einen Teil über die GPU rendern und damit erhebliche Performanceverbesserungen bringen werden.
			
	
	
\chapter{Firewall Testing}
	Beim Firewall Testing ging es darum, herauszufinden, in welchen Netzwerkumgebungen WebRTC-Kommunikation eingesetzt werden kann, und in welchen nicht.
	
	
	\section{Signaling}
		Für das Signaling können verschiedene Protokolle eingesetzt werden. Bevorzugt HTTP oder WebSockets. Davon hängt auch ab, ob die Pakete von Firewalls ausgefiltert werden, oder nicht.
		
		WebSocket Pakete sind insbesondere älteren Firewalls noch nicht bekannt und werden von diesen verworfen. Dies betrifft jedoch vor Allem Unternehmensnetzwerke.
		In den Netzwerken der Schweizer Provider sowie im Netzwerk der HSR konnten wir dies nie beobachten.
		Mit VPN könnte man das Problem umgehen, sofern VPN im Netzwerk zugelassen ist.
		
		Wird HTTP für das Signaling eingesetzt, kommt es zu keinen Paketverwürfen, da HTTP von allen Firewalls erlaubt wird.
		In Unternehmen ist es vorstellbar, das unverschlüsselte HTTP-Pakete analysiert und aufgrund der enthaltenen SDP blockiert werden. Durch den Einsatz von HTTPS lässt sich dies jedoch lösen.
		
		Werden Signaling Pakete blockiert, so erhalten die Peers nie Verbindungsanfragen von anderen Peers.
		

	\section{P2P Connection}
		Für den Aufbau der Peer-to-Peer Connection ermitteln beide Clients über einen
		STUN-Service ihre externe IP, tauschen diese über den Signalingchannel
		mittels SDP\footnote{Session Description Protocol} aus und verbinden sich
		direkt P2P um Parameter für die Verschlüsselung auszutauschen. Für diesen
		Austausch wird ebenfalls das UDP basierte STUN-Protokoll verwendet.
		
		In starkt reglementierten Umgebungen werden unter Umständen diese STUN-Pakete
		geblockt. Dadurch kann keine Verbindung zwischen den Clients aufgebaut werden und die Applikation wird nach dem eingestellten Timeout den Verbindungsaufbau abbrechen. 
		Um durch eine solch restriktive Firewall hindurchzukommen kann ein VPN Client eingesetzt werden, vorausgesetzt VPN ist zugelassen.
		
		In Netzwerkumgebungen wie sie Heimanwender oder kleinere Firmen verwenden sollten STUN-Pakete nicht blockiert werden. Auch im Mobilfunk konten wir dies bisher nicht beobachten.
	
		Detailierte Firewalltests: Siehe Anhang \ref{firewalltests}
		
		