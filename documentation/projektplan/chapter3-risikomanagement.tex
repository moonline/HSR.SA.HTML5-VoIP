\chapter{Risikomanagement}
\section{Risiken}

% example
\begin{table}[h!]
	\centering
	%Die Spalten werden aufgeteilt auf 0.3 resp. 0.7-fache Textbreite
	\begin{tabular}{|p{0.3\textwidth} | p{0.7\textwidth} |}
	\hline	
	Risiko-ID & 1 \\
	\hline
	Titel & Implementierungshindernis \\
	Beschreibung & Aufgrund einer nicht bedachten Schwierigkeit, verzögert sich die Entwicklung des Programms \\
	max. Schaden	& 10h \\
	Eintrittswahrscheinlichkeit & 0.4 \\
	Gewichteter Schaden	& 4h \\
	Vorbeugung	& Sorgfältige Abklärungen im Vorfeld der Implementierung \\
	Massnahmen	& Zusätzliche Entwicklungszeit, um das Problem zu lösen \\
	\hline
	\end{tabular}
\end{table}


\clearpage	

\section{Umgang mit Risiken}
