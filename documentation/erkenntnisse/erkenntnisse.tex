\chapter{Erkenntnisse}
	\section{Media Streaming}
		\subsection{Verbindungsabbruch}
			Bricht die Verbindung ab, so bleibt die Video- und Audiowiedergabe stehen, bis die Verbindung wieder da ist und die UDP Pakete wieder ankommen.
		
		\subsection{Downsampling}
	 		Der Mediastream wird automatisch mit Minimalqualität gestartet und abhängig von Prozessorleistung und Bandbreite hochgefahren bis zu einer Datenrate von etwa 300MiB/s.
	 		
	 	\subsection{Resolution Properties}
	 		Obwohl die Spezifikation für getUserMedia eine Möglichkeit zur Festlegung der gewünschten Auflösung definiert, ist dies in den Browsern bisher nicht implementiert.
	 		
	 	\subsection{Audio/Video attach/detach}
	 		Hinzufügen und Entfernen von Streams sollte eigentlich möglich sein. Die Browser implementieren dies bisher jedoch nicht. Gemäss dem Mozilla Blog soll dieses Feature jedoch demnächst umgesetzt werden.	 	
	 	
	 \section{Peerconnection}
	 	Eine Peer Connection kann nur durch den Austausch von SDP Descriptions aufgebaut werden. Innerhalb der SDP werden Parameter für die Peerconnection sowie IP und allenfalls Endpunktkandidaten übertragen. 
	 	
	 	\subsection{Endpunktkandidaten}
	 	Je nach Implementation übertragen die Browser die Candidates einzeln. Ist die Verbindung noch nicht initialisiert worden, so müssen diese zwischengespeichert werden bis die Verbindung so weit ist.

		\subsection{Verschlüsselung}
	 	
	 
	 \section{Filezugriff}
	 	Dateizugriff ist in Javascript nur über einen User Event möglich (Upload Field oder Drag/Drop). Daher ist es nicht möglich, Adressbücher auf die Festplatte abzulegen und wieder zu öffnen ohne den User dazu aufzufordern.
	 	
	 
	 \section{Offline Einschränkungen}
	 	\section{Stun Services}
	 	
	 	
	 		
	 \section{SIP}
	 	\subsection{Websockets}
	 		Obwohl der OpenSource SIP Server Kamailio WebSocket als Feature aufführt, ist die Implementation noch nicht vollständig, sodass SIP Kommunikation über WebSockets noch unmöglich ist.


	\section{Tipps \& Tricks für WebRTC Applikationsentwickler}
		