\chapter{Erkenntnisse}
	\section{Media Streaming}
		\subsection{Verbindungsabbruch}
			Bricht die Verbindung ab, so bleibt die Video- und Audiowiedergabe stehen, bis die Verbindung wieder da ist und die UDP Pakete wieder ankommen.
		
		\subsection{Downsampling}
	 		Der Mediastream wird automatisch mit Minimalqualität gestartet und abhängig
	 		von Prozessorleistung und Bandbreite hochgefahren, bis zu einer Datenrate von etwa 300MiB/s.
	 		
	 	\subsection{Resolution Properties}
	 		Obwohl die Spezifikation für getUserMedia eine Möglichkeit zur Festlegung der gewünschten Auflösung definiert, ist dies in den Browsern bisher nicht implementiert.
	 		
	 	\subsection{Audio/Video attach/detach}
	 		Hinzufügen und Entfernen von Streams sollte eigentlich möglich sein. Die
	 		Browser implementieren dies bisher jedoch nicht. Gemäss dem Mozilla-Blog soll dieses Feature jedoch demnächst umgesetzt werden.
	 	
	 \section{Peerconnection}
	 	Eine Peer-Connection kann nur durch den Austausch von SDP-Descriptions
	 	aufgebaut werden. Innerhalb der SDP werden Parameter für die Peer-Connection
	 	sowie IP und allenfalls Endpunktkandidaten übertragen.
	 	
	 	\subsection{Endpunktkandidaten}
	 	Je nach Implementation übertragen die Browser die Candidates einzeln. Ist die
	 	Verbindung noch nicht initialisiert worden, so müssen diese
	 	zwischengespeichert werden, bis die Verbindung bereit ist.

		\subsection{Verschlüsselung}
			Der übertragene Stream ist standardmässig verschlüsselt. Die Verschlüsselung ist Teil des Standards und kann nicht abgeschaltet werden.
	 	
	 
	 \section{Filezugriff}
	 	Dateizugriff ist in JavaScript nur über einen User-Event möglich (Upload
	 	Field oder Drag/Drop). Daher ist es nicht möglich, Adressbücher auf die
	 	Festplatte abzulegen und wieder zu öffnen, ohne den User dazu aufzufordern.
	 	
	 
	 \section{Offline-Einschränkungen}
	 	\subsection{Stun Services}
			Ohne eigener Stun Service ist eine Internet Verbindung zwingend. Für Telefonie in einem abgeschotteten Netz wird ein eigener Stun Service benötigt. Es ist nicht möglich auf die Stun Services zu verzichten und die Adresse manuel zu setzen, weil das Stun Protokoll auch den P2P Verbindungsaufbau übernimmt.
	 	
	 		
	 \section{SIP}
	 	\subsection{WebSockets}
	 		Obwohl der OpenSource-SIP-Server Kamailio WebSocket als Feature aufführt,
	 		ist die Implementation noch nicht vollständig, sodass SIP-Kommunikation über
	 		WebSockets noch unmöglich ist.
	 		
	\section{Browserunterschiede}
		\subsection{Ice Candidates}
			Firefox packt die Ice Candidates\footnote{Vorschläge für Ports} direkt in die Session Description ein. Chrome schickt sie einzeln. Da Chrome nicht auf die Reihenfolge achtet kann es passieren, das Ice Candidates vor einer Offer ankommen. Aus diesem Grund müssen ankommende Ice Candidates in einer Queue gebuffert werden, bis eine Offer ankommt.
		
		\subsection{DataChannel}
			Firefox unterstützt DataChannel\footnote{P2P-Datenkanal parallel zum Video-
			und Audiostream} bereits vollständig. Chrome unterstützt ihn nur im Modus
			"`reliable: false"'. Setzt man ihn auf "`reliable: true"', kommen die Events
			nicht zuverlässig an.
			
		\subsection{Media Access}
			Firefox erlaubt mehrfachen Kameraacces und teilt sich die Kamera auch problemlos mit andern Prozessen. So ist es im Firefox möglich, von einem Tab in einen andern anzurufen. Chrome will die Kamera komplett für sich beanspruchen und wirf eine Fehlermeldung, falls sie bereits von einem Andern Prozess genutzt wird.
			
		\subsection{Schnittstellenbezeichnung}
			Die Schnittstellen unterstützen beide Browser erst mit Prefix. Die offizielle Bezeichnung wird noch nicht unterstützt. Adapter.js wurde dafür entwickelt und mappt die offiziellen Schnittstellenbezeichnungen auf die mit Prefix, sodass in der Applikation trotzdem mit den offiziellen gearbeitet werden kann und auch Kompatibilität besteht, sobald die Browser auf die definitiven Schnittstellen umstellen.


	\section{Tipps \& Tricks für WebRTC-Applikationsentwickler}
		\subsection{Verbindungsaufbau}
			Ice Candidates müssen einer RTCPeerConnection attatched werden, bevor eine Offer oder eine Answer generiert wird. Ansonsten