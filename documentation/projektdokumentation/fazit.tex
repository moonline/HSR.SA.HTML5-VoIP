\chapter{Schlussfolgerungen}
	\section{Was wurde erreicht}
	Es ist gelungen, mit JavaScript eine Video-Over-IP Applikation zu entwickeln, die ohne Serverkomponenten auskommt.
	
	Die Applikation benutzt die neuen WebRTC Schnittstellen ``MediaStream"', ``RTCPeerConnection"' und ``RTCDataChannel"'. Daneben werden ``LocalStorage"' für die Speicherung der Kontaktbücher und der Benutzeraccounts sowie ``FileReader"' für den Import der Kontaktbücher verwendet.
	
	Die Benutzeroberfläche wurde responsive gestaltet um auf dem Desktop wie auf Mobilgeräten eine angenehme Benutzung zu ermöglichen.
	
	
	\section{Offene Punkte}
	Internationalisierung, SIP Support, CSV Contactbook Adapter wurden bereits im dritten Viertel des Projektes aus den Featurelist entfernt, da sie tieferer Priorität waren und ein umfangreiches Refactoring der Core-Funktionalität wichtiger war.
	
	Auf eine Authentifizierung beim Channel und entsprechende Mechanismen im QueueServer wurde verzichtet, da es sich nur um eine Referenzimplementation handelt und jeder Channel dies selbst handeln kann.
	
	
	\section{Bugs und technische Probleme}
	Es ist nicht gelungen, SIP als Signaling-Protokoll einzusetzen, da die SIP-Server Websockets erst seit kurzem unterstützen und die Schnittstelle im verwendeten SIP-Server Kamailo noch nicht stabil ist. Websockets sind zwingend notwendig, 