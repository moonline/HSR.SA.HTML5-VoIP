\chapter{Risikomanagement}
\label{risiken} 
\section{Risiken}

%Die Spalten werden aufgeteilt auf 0.3 resp. 0.7-fache Textbreite
\noindent
\begin{tabular}{|p{0.3\textwidth} | p{0.7\textwidth} |}
	\hline	
	Risiko-ID & 1 \\
	\hline
	Titel & Implementierungshindernis \\
	Beschreibung & Aufgrund einer nicht bedachten Schwierigkeit verzögert sich die
	Entwicklung des Programms. \\
	max. Schaden	& 10h \\
	Eintrittswahrscheinlichkeit & 0.4 \\
	Gewichteter Schaden	& 4h \\
	Vorbeugung	& Sorgfältige Abklärungen im Vorfeld der Implementierung. \\
	Massnahmen	& Zusätzliche Entwicklungszeit, um das Problem zu lösen. \\
	\hline
\end{tabular}
\hspace{0.5cm}
\newline

\noindent
\begin{tabular}{|p{0.3\textwidth} | p{0.7\textwidth} |}
	\hline	
	Risiko-ID & 2 \\
	\hline
	Titel & Verbindungsverlust \\
	Beschreibung & Kurze Unterbrüche in der Verbindung könnten die ganze Session
	beenden. \\
	max. Schaden	& 8h \\
	Eintrittswahrscheinlichkeit & 0.9 \\
	Gewichteter Schaden	& 7.2h \\
	Vorbeugung	& Einarbeitung in SIP Connection Management. \\
	Massnahmen	& Session bei kleinen Unterbrüchen aufrecht erhalten und einen
	schnellen Reconnect einrichten. \\
	\hline
\end{tabular}
\hspace{0.5cm}
\newline
	
	
\noindent
\begin{tabular}{|p{0.3\textwidth} | p{0.7\textwidth} |}
	\hline	
	Risiko-ID & 3 \\
	\hline
	Titel & Browserperformance \\
	Beschreibung & Video und Audio müssen flüssig wiedergegeben werden. \\
	max. Schaden	& 3h \\
	Eintrittswahrscheinlichkeit & 0.1 \\
	Gewichteter Schaden	& 0.3h \\
	Vorbeugung	& Performance-Tests mit Prototypen. \\
	Massnahmen	& Skalierung der Videoqualität, notfalls Priorisierung auf Audio. \\
	\hline
\end{tabular}
\hspace{0.5cm}
\newline
	
	
\noindent
\begin{tabular}{|p{0.3\textwidth} | p{0.7\textwidth} |}
	\hline	
	Risiko-ID & 4 \\
	\hline
	Titel & Langsame Verbindung \\
	Beschreibung & Bei langsamen Verbindungen kann es zu Problemen in der
	Wiedergabe führen. \\
	max. Schaden	& 8h \\
	Eintrittswahrscheinlichkeit & 0.8 \\
	Gewichteter Schaden	& 6.4h \\
	Vorbeugung	& Performance-Tests mit Prototyp und künstlichem Traffic. \\
	Massnahmen	& Skalierung der Videoqualität, notfalls Priorisierung auf Audio. \\
	\hline
\end{tabular}
\hspace{0.5cm}
\newline
	
	
\noindent
\begin{tabular}{|p{0.3\textwidth} | p{0.7\textwidth} |}
	\hline	
	Risiko-ID & 5 \\
	\hline
	Titel & Konferenzschaltung \\
	Beschreibung & Hohe Teilnehmerzahl führt zu hoher Anzahl Verbindungen. \\
	max. Schaden	& 1h \\
	Eintrittswahrscheinlichkeit & 1.0 \\
	Gewichteter Schaden	& 1h \\
	Vorbeugung	& Performance-Tests mit Prototypen. \\
	Massnahmen	& Maximale Anzahl Teilnehmer festlegen. \\
	\hline
\end{tabular}
\hspace{0.5cm}
\newline

	
\noindent
\begin{tabular}{|p{0.3\textwidth} | p{0.7\textwidth} |}
	\hline	
	Risiko-ID & 6 \\
	\hline
	Titel & Kompetenzmangel SIP \\
	Beschreibung & Erfahrung mit SIP mangelt. Unbekannte Probleme möglich. \\
	max. Schaden	& 14h \\
	Eintrittswahrscheinlichkeit & 0.7 \\
	Gewichteter Schaden	& 9.8h \\
	Vorbeugung	& Frühe Einarbeitung in SIP. \\
	Massnahmen	& Zusätzliche Entwicklungszeit. \\
	\hline
\end{tabular}
\hspace{0.5cm}
\newline


\noindent
\begin{tabular}{|p{0.3\textwidth} | p{0.7\textwidth} |}
	\hline	
	Risiko-ID & 7 \\
	\hline
	Titel & NAT \& Firewall Traversal \\
	Beschreibung & NATs und Firewalls lassen sich nicht ohne Weiteres durchdringen.
	\\
	max. Schaden	& 20h \\
	Eintrittswahrscheinlichkeit & 0.25 \\
	Gewichteter Schaden	& 5h \\
	Vorbeugung	& Früh Verbindungstests mit NATs und Firewalls durchführen. \\
	Massnahmen	& Zusätzliche Entwicklungszeit investieren zur Unterstützung von
	Proxyservern und Tunneling-Mechanismen. \\
	\hline
\end{tabular}
\hspace{0.5cm}
\newline


\noindent
\begin{tabular}{|p{0.3\textwidth} | p{0.7\textwidth} |}
	\hline	
	Risiko-ID & 8 \\
	\hline
	Titel & Verschlüsselte Verbindung \\
	Beschreibung & Das Verschlüsseln der P2P-Verbindung und des Server-Connects
	gestalten sich schwieriger als gedacht. \\
	max. Schaden	& 16h \\
	Eintrittswahrscheinlichkeit & 0.5 \\
	Gewichteter Schaden	& 8h \\
	Vorbeugung	& Früh Möglichkeiten zur Verschlüsselung und deren
	Implementierungsaufwand analysieren. \\
	Massnahmen	& Zusätzliche Entwicklungszeit investieren zur Umsetzung des
	Verschlüsselungsmechanismus. \\
	\hline
\end{tabular}
\hspace{0.5cm}
\newline


\noindent
\begin{tabular}{|p{0.3\textwidth} | p{0.7\textwidth} |}
	\hline
	Risiko-ID & 9 \\
	\hline
	Titel & Social Risks \\
	Beschreibung & Die Projektmitglieder sind aufgrund anderer Projekte zu stark
	absorbiert und können zu wenig Zeit für das Projekt aufwenden. \\
	max. Schaden & 25h \\
	Eintrittswahrscheinlichkeit & 0.25 \\
	Gewichteter Schaden	& 6h \\
	Vorbeugung	& Externe Projekte in der Iterationsplanung berücksichtigen. \\
	Massnahmen	& Projektmitglieder opfern ihre Freizeit und trinken mehr
	koffeinhaltige, flügelverleihende Energy Drinks.
	\\
	\hline
\end{tabular}
\hspace{0.5cm}
\newline


\section{Umgang mit Risiken}
Um die Riskien möglichst kalkulierbar zu halten, ist es für dieses Projekt
entscheidend, möglichst früh Evaluationen durchzuführen. Besonders die frühe Einarbeitung in SIP sowie die Bereitstellung eines ersten
Prototypen mit SIP-Verbindung muss gewährleistet werden. Sobald Prototypen
vorhanden sind, müssen erste Performance-Test durchgeführt werden. Weiterhin
sollten die Risiken laufend neu beurteilt werden.
