%Pakete;
%A4, Report, 12pt
\documentclass[ngerman,a4paper,12pt]{scrreprt}
\usepackage[a4paper, right=20mm, left=20mm,top=30mm, bottom=30mm, marginparsep=5mm, marginparwidth=5mm, headheight=7mm, headsep=15mm,footskip=15mm]{geometry}

%Papierausrichtungen
\usepackage{pdflscape}
\usepackage{lscape}

%Deutsche Umlaute, Schriftart, Deutsche Bezeichnungen
\usepackage[utf8]{inputenc}
\usepackage[T1]{fontenc}
\usepackage[ngerman]{babel}

%quellcode
\usepackage{listings}

%tabellen
\usepackage{tabularx}

%listen und aufzählungen
\usepackage{paralist}

%farben
\usepackage[svgnames,table,hyperref]{xcolor}

%font
\usepackage{helvet}
\renewcommand{\familydefault}{\sfdefault}

%Abkürzungsverzeichnisse
\usepackage[printonlyused]{acronym}

%Bilder
\usepackage{graphicx} %Bilder
\usepackage{float}	  %"Floating" Objects, Bilder, Tabellen...

%Dokumenteigenschaften
\title{Anforderungsspezifikationen WebRTC VoIP Applikation}
\author{Tobias Blaser, Beat Gutzwiller}
\providecommand{\teacher}{Luc Bläser}
\providecommand{\room}{1.267}
\providecommand{\versionnumber}{1.0}
\date{\today{}, Rapperswil}


%Kopf- /Fusszeile
\usepackage{fancyhdr}
\usepackage{lastpage}

\pagestyle{fancy}
\fancyhf{} %alle Kopf- und Fußzeilenfelder bereinigen
\fancyhead[L]{Semester Arbeit} %Kopfzeile links
\fancyhead[C]{WebRTC VoIP Applikation} %Kopfzeile mitte
\fancyhead[R]{Seite \thepage/\pageref{LastPage}} %Kopfzeile rechts
\renewcommand{\headrulewidth}{0.4pt} %obere Trennlinie
\fancyfoot[L]{\jobname} %Fusszeile links
\fancyfoot[C]{Version: \versionnumber} %Fusszeile mitte
\fancyfoot[R]{\today{}} %Fusszeile rechts
\renewcommand{\footrulewidth}{0.4pt} %untere Trennlinie

%Kopf-/ Fusszeile auf chapter page
\fancypagestyle{plain} {
	\fancyhf{} %alle Kopf- und Fußzeilenfelder bereinigen
	\fancyhead[L]{Semester Arbeit} %Kopfzeile links
	\fancyhead[C]{WebRTC VoIP Applikation} %Kopfzeile mitte
	\fancyhead[R]{Seite \thepage/\pageref{LastPage}} %Kopfzeile rechts
	\renewcommand{\headrulewidth}{0.4pt} %obere Trennlinie
	\fancyfoot[L]{\jobname} %Fusszeile links
	\fancyfoot[C]{Version: \versionnumber} %Fusszeile mitte
	\fancyfoot[R]{\today{}} %Fusszeile rechts
	\renewcommand{\footrulewidth}{0.4pt} %untere Trennlinie
}

\usepackage{changepage}

%links, verlinktes Inhaltsverzeichnis, PDF Inhaltsverzeichnis
\usepackage[bookmarks=true,
bookmarksopen=true,
bookmarksnumbered=true,
breaklinks=true,
colorlinks=true,
linkcolor=black,
anchorcolor=black,
citecolor=black,
filecolor=black,
menucolor=black,
pagecolor=black,
urlcolor=black
]{hyperref} % Paket muss unbedingt als letzes eingebunden werden!

\begin{document}

%Titel und Inhaltsverzeichnis
\thispagestyle{empty}
\begin{titlepage}
	\begin{center}

	\vspace*{40mm}
	
	\begin{figure}[htp]
		\centering
		%\includegraphics[scale=0.60]{../../logo.png}
	\end{figure}		
	\vspace*{20mm}
	
	{\fontsize{40}{48} \selectfont WebRTC VoIP Applikation \\[10mm]}
	{\fontsize{40}{48} \selectfont Anforderungsspezifikationen \\[5mm]}	
	\vspace*{20mm}
	Tobias Blaser, Beat Gutzwiller

\end{center}
\end{titlepage}
\clearpage

\chapter*{Änderungsnachweis}
\begin{tabularx}{\textwidth}{|cXlr|} % Versionstabelle, Rahmen links und rechts
		\hline
		\textbf{Version} & \textbf{Änderung} & \textbf{Autor} & \textbf{Datum}\\
		\hline
		1.0 & Dokumentenentwurf & Tobias Blaser & 17.09.13\\
		\hline
\end{tabularx}

% Inhaltsverzeichnis
\tableofcontents

\chapter{Einführung}

\section{Beschreibung}
Dieses Dokument beinhaltet die Anforderungen für das Projekt \project.

\section{Gültigkeitsbereich}
Dieses Dokument ist für das gesammte Projekt \project gültig.

\chapter{Anforderungen}
\section{Allgemeine Beschreibung}

\subsection{Produktfunktion}
Anwender sollen Peer-to-Peer Audio- und Videotelefonie durchführen können, ohne
dazu eine zusätzliche Software installieren zu müssen. Sie sollen die Telefonate
direkt in ihrem Browser durchführen können und ihre Kontaktdaten über
verschiedene Schnittstellen laden können.

\subsection{Benutzer-Charakteristik}
Zielgruppe der "`WebRTC VoIP Applikation"' sind gewöhnliche Anwender ohne
technische Kenntnisse.

\subsection{Einschränkungen}
Voraussetzung für die Nutzung der "`WebRTC VoIP Applikation"' ist ein moderner Browser mit
WebRTC-Unterstützung (aktuell Firefox, Chrome und Opera, Stand 18.12.13).


\section{Use Cases}

	\subsection{Use-Case-Übersicht}
	\begin{itemize}
		\item Anrufen
		\item Telefonbuch importieren
	\end{itemize}

	\subsection{Use-Case-Beschreibungen}

		\subsubsection{Anrufen}
		Voraussetzungen: Der Benutzer hat eine Kamera und ein Mikrofon angeschlossen,
		ist angemeldet und hat einen Account beim gleichen Channel wie die Person, die
		er anrufen möchte.\newline

		Der Benutzer ruft einen anderen Benutzer an. Ist der andere Benutzer
		erreichbar und verwendet einen kompatiblen Client (WebRTC), so wird eine gemeinsame Audio- oder Videosession gestartet.
		Am Ende des Telefonates beendet der Benutzer die Session und die Verbindung
		wird beendet.

		Während dem Telefonat wird die Auflösung der Verbindungsqualität angepasst.

		\subsubsection{Telefonbuch importieren}
		Voraussetzungen: Der Benutzer ist angemeldet und befindet sich in der
		Kontaktbuchansicht.\newline
		
		Der Benutzer wählt eine Quelle aus und importiert die entsprechenden Daten. Die Telefonbucheinträge werden dem Benutzer nach Quelle getrennt angezeigt.

\section{Weitere Anforderungen}

\subsection{Qualitätsmerkmale}
Die Applikation soll intuitiv bedienbar sein und die Video- und Audio-Auflösung
soll abhängig von der Verbindungsqualität skaliert werden.


\subsubsection{Functionality}
\begin{itemize}
	%\item[Korrektheit:] 
	%\item[Angemessenheit:] 
	\item[Interoperabilität:] Die Applikation soll auf existierende Standards setzen und wo eigene Standards nötig sind, diese offen dokumentieren.
	\item[Sicherheit:] Die Audio- und Videokommunikation sowie die Verbindung zum Server soll nach Möglichkeit verschlüsselt erfolgen.
\end{itemize}

%\subsubsection{Efficiency}
%\begin{itemize}
%	\item[Wirtschaftlichkeit:] 
%	\item[Zeitverhalten:] 
%\end{itemize}

\subsubsection{Usability}
\begin{itemize}
	\item[Verständlichkeit:] Die Benutzeroberfläche soll selbsterklärend aufgebaut sein. Knöpfe und Labels sollen sprechend bezeichnet sein.
	%\item[Bedienbarkeit:] 
	\item[Robustheit:] Die Oberfläche soll nicht hängen bleiben bei länger dauernden Operationen und auf Fehler angemessen reagieren.
\end{itemize}

%\subsubsection{Reliability}
%\begin{itemize}
%	\item[Reife:] RC 1
%	\item[Fehlertoleranz:] 
%	\item[Wiederherstellbarkeit:] 
%\end{itemize}

\subsubsection{Portability}
\begin{itemize}
	\item[Anpassbarkeit:] Die Applikation soll so gestaltet werden, das sich die Unterstützung für weitere Technologien einfach hinzufügen lässt.
	\item[Installierbarkeit:] Die Applikation soll auf jedem WebRTC fähigen modernen Browser laufen, der sich an die Standard Darfts hält.
	\item[Austauschbarkeit:] Die Oberfläche soll durch eigene Styles einfach umgestaltet und angepasst werden können.
\end{itemize}

%\subsubsection{Maintainability}
%\begin{itemize}
%	\item[Analysierbarkeit:] 
%\end{itemize}

\subsection{Schnittstellen}

\subsubsection{Signalling-Schnittstelle}
Die Applikation soll als Signalling-Technologie einen offenen Standard (zum
Beispiel SIP oder XAMPP) unterstützen.

\subsubsection{Telefonbuchschnittstelle}
Die Telefonbuchschnittstelle soll offene und verbreitete Standardformate
unterstützen (beispielsweise vCard, CSV, JSON …).


\end{document}
