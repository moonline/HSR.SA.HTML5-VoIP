\chapter{Abstract}

Ziel der Studienarbeit ``A Practical JavaScript-Only Video-Over-IP Communication Plattform'' war die Entwicklung einer JavaScript Applikation zur Audio- und Videokommunikation, die ausschliesslich im Browser läuft. Benutzer müssen keine Software auf ihrem Endgerät installieren, um die Applikation nutzen zu können. Die Plattform basiert auf standardisierter Webtechnologie, sodass jedes unterstützende und genügend leistungsfähige Gerät die Applikation nutzen kann.
Teil der Arbeit waren die Erforschung des aktuellen Standes der Technologie und der notwenigen Schnittstellen, die Umsetzung von Referenzimplementationen der eigenen Schnittstellen sowie eine Analsyse der Performance.

Die Unterstützung durch die Browser ist noch sehr unterschiedlich. Nur neuere Chrome-, Firefox- und Opera-Versionen unterstützen die Schnittstellen, wobei die Unterstützung experimentell ist und entsprechend variiert. Mediacodierung und Verschlüsselung verbrauchen auch noch spürbar viel Rechenleistung. Die automatische Datenratenskalierung wird bereits durch alle getesteten Browser unterstützt.

Die Technologien hinter den JavaScript-Schnittstellen setzen Protokolle wie STUN oder ICE ein, die von stark restriktiven Firewalls blockiert werden. In gewöhnlichen Netzwerkumgebungen sowie im Mobilfunknetz stellt dies jedoch keine Probleme dar. 

Wir haben es erfolgreich geschafft, eine JavaScript-basierte Videokommunikationsplattform mit dem aktuellen Stand der Browsertechnologie aufzubauen. Alle browserseitig notwendigen Schnittstellen existieren und funktionieren bereits. Trotzdem ist deren Umsetzung noch experimentell, was sich vor allem durch den Resourcenverbrauch bemerkbar macht.

Die Benutzeroberfläche wurde sowohl für Desktop wie Mobilgeräte konzipiert und übersichtlich gestaltet. Unterstützt werden Benutzerkonten sowie Import von Kontaktdaten.
Über Schnittstellen kann die Applikation um eigene Kontaktdatenformate und Verbindungsmechanismen erweitert werden.