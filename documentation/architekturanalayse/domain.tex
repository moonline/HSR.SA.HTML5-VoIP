\begin{landscape}
\section{Domainmodel}
	Im folgenden Diagramm sind die wichtigsten konzeptionellen Klassen und ihre Beziehungen untereinander aufgeführt.
	\begin{figure}[H]
		\centering
		\includegraphics[width=1.2\textwidth]{../architekturanalayse/img/domain.png}
		%\caption{Strukturdiagramm JS VoIP App}
	\end{figure}
	Herzstück der Applikation sind die Channels und die Connection sowie das Contactbookmanagement.

\end{landscape}

\section{Channel}
	Die Channels laufen im Hintergrund und verbinden den User mit einem oder
	mehreren Signaling-Servern. Sie werden beim Login des Benutzers gestartet und
	müssen entsprechend ein Channel-Interface implementieren. So laufen z.\,B. ein
	SIP Channel und ein XHR Channel, abhängig davon, was der Benutzer für Accounts
	besitzt.
	\begin{figure}[H]
		\centering
		\includegraphics[width=\textwidth]{../architekturanalayse/img/channel.png}
		%\caption{Strukturdiagramm JS VoIP App}
	\end{figure}
	
	Die interne Implementation des Channels ist dem entsprechenden Entwickler
	überlassen. Als Referenzimplementation wurde ein Channel über XHR und ein Channel über WebSockets umgesetzt.
	
	Beim XHR-Channel wird mit dem Starten des Channels ein Polling auf den
	Queue-Server gestartet. Sobald dieser eine Antwort zurückliefert, werden die Listener benachrichtigt.
	
	Beim WebSocket-Channel wird eine WebSocket-Verbindung geöffnet, die
	anschliessend offen bleibt. Sobald der Queue-Server einen neue Nachricht hat, erhält der Channel diese vom Server und kann die Listener benachrichtigen.
	
	Sowohl der XHR-Channel wie der WebSocket-Channel lassen sich verschlüsselt
	nutzen, indem beim XHR-Channel über HTTPS\footnote{TLS geschützte HTTP Verbindung} zugegriffen wird und beim WebSocket-Channel über WSS\footnote{Secure WebSocket, TLS geschützte WebSocket-Verbindung}.
	

\section{Adressbook}
	Die Kontaktverwaltung übernimmt der ContactbookManager. Er importiert
	Kontaktbücher anhand der Konfiguration von den Quellen "`Datei"', "`Ordner"'
	oder "`Online"'. Für die format\-spezifischen Importprozesse sind die
	jeweiligen Kontaktbücher zuständig. Dazu müssen alle Kontaktbücher das Addressbook-Interface umsetzen.
	\begin{figure}[H]
		\centering
		\includegraphics[width=\textwidth]{../architekturanalayse/img/addressbook.png}
		%\caption{Strukturdiagramm JS VoIP App}
	\end{figure}

	Kontaktbücher werden im LocalStorage gespeichert und nach einem Neustart wieder
	geladen. Da im LocalStorage nur serialisierte Daten gespeichert werden können,
	müssen neue Kontaktbuchobjekte erstellt und die Daten injected werden.


\section{Connection}
	Die Connection übernimmt das komplette Handling der P2P-Verbindung inklusive Verbindungsauf- und -abbau. Sie greift über den Host auf die lokale Kamera zu und sendet und empfängt über einen Channel "`Offers"' und "`Answers/Replies"'.
	\begin{figure}[hp]
		\centering
		\includegraphics[width=\textwidth]{../architekturanalayse/img/connection.png}
		%\caption{Strukturdiagramm JS VoIP App}
	\end{figure}
		
	\subsection{Verbindungsaufbau}
		Die Kommunikation zwischen zwei Clients, einem STUN-Service und einem
		Signaling-Service wird im folgenden Diagramm dargestellt.
		\begin{figure}[H]
			\centering
			\includegraphics[width=\textwidth]{../architekturanalayse/img/callDiagramm.png}
			\label{img:deployment}
			\caption{Clientkommunikation}
		\end{figure}
		Die Clients beziehen vom STUN-Service ihre extern sichtbare IP. Nach dem
		Austauschen der SDP\footnote{Session Description Protocol\cite{IETF-SDP-RFC}}
		über den Signaling Channel versuchen die Peers sich gegenseitig direkt zu
		erreichen mit STUN Messages. Dazu verwenden sie die Ports, die sie von den
		Candidates\footnote{Endpunktkandidaten (Ports) für DataChannel, Audio- und Videostream} des SDP erhalten haben.
		War dies erfolgreich, wird die PeerConnection aufgebaut und die Datenströme
		sind verbunden. Die Peers kommunizieren nun direkt und benötigen den
		Sig\-naling Service nicht mehr.
				
		\begin{figure}[H]
			\centering
			\includegraphics[width=\linewidth]{../architekturanalayse/img/connectionState.png}
			\label{img:deployment}
			\caption{Applikationszustände}
		\end{figure}
		Beim Verbindungsaufbau nimmt jeder Client entweder die Rolle "`Caller"' oder
		"`Callee"' ein. Erlaubt der Benutzer Zugriff auf die Kamera, so wird die
		Verbindung aufgebaut und bleibt bestehen, bis einer der Clients die Verbindung
		mit einem "`Bye"' beendet.
		
		Ungültige Messages werden ignoriert. "`Bye"' führt in jedem Zustand zu einem
		Abbruch der Verbindung bzw. des Aufbaus, vorausgesetzt das "`Bye"'
		kommt vom anderen Kommunikationspartner.
	
	\subsection{Verbindungsabbau}
		Für den Verbindungsabbau teilen sich die Peers dies mit und bauen anschliessend die Verbindung ab. Der Verbindungsabbau kann wahlweise über den Signaling-Server oder direkt über den Datachannel der PeerConnection gesendet werden. Wurde die PeerConnection als unzuverlässig konfiguriert, so sind Bestätigungsnachrichten sinnvoll, damit kein Client die Verbindung offen behält, obwohl der andere Peer bereits die Verbindung abgebaut hat.
	
	\subsection{STUN-Service}
		STUN\footnote{Session Traversal Utilities for NAT \cite{IETF-STUN-RFC}}-Services sind zwingend
		notwendig, um die nach aussen sichtbare IP-Adresse des Clients herauszufinden.
		Neben der Möglichkeit, selbst einen STUN-Service aufzusetzen, gibt es frei
		verfügbare STUN-Services, beispielsweise von Mozilla oder Google.

		Für ein abgeschottetes Netz ist ein eigener STUN-Server zwingend notwendig.
		Für unseren Fall reichen die frei verfügbaren. Durch anpassen der
		Konfiguration ist es möglich, den Server festzulegen.

		Im Netz gibt es fertig konfigurierte virtuelle Maschinen mit einsatzbereitem
		STUN-Service\footnote{Mozilla, stun-vm \cite{Mozilla-STUN-VM}}.
		
			\subsection{SIP-Proxy vs. SIP-Server mit WebSockets}
		Unterstützt ein SIP-Server keine WebSockets, so kann ein SIP/WebSocket-Proxy
		dazwischen geschaltet werden. Die Konfiguration eines SIP-Proxies ist ähnlich
		aufwändig wie die Installation eines Kamailio
		\footnote{Kamailio SIP Server Project, \cite{Kamailio-Project}}-SIP-Servers, der in der neusten Version WebSockets unterstützt.

	\subsection{Sicherheit}
		WebRTC-Streams können nur verschlüsselt übertragen werden. Es gibt keine
		Möglichkeit, die Verschlüsselung abzuschalten. Dies führt zwar zu einem
		erhöhten Rechenbedarf auf dem Client, garantiert jedoch eine verschlüsselte
		Verbindung unabhängig von den Vorlieben des Entwicklers.
		Die Verschlüsselung erfolg über DTLS-SRTP\footnote{Datagram Transport Layer Security \cite{IETF-DTLS-RFC}} 
		keyings\footnote{Adam Roach, WebRTC: Security and Confidentiality \cite{AdamRoach-WebRTC-Security}}. 
		
		Die Verschlüsselung des Signaling-Channels ist abhängig von der eingesetzten
		Technologie. Werden z. B. Secure Websockets oder XHR über HTTPS eingesetzt so ist die Kommunikation verschlüsselt und nicht abgreifbar.
		
		Obwohl die Daten sicher übertragen werden, kann ein Provider oder eine anderweitige Zwischenstelle Metadaten darüber sammeln, wer mit wem telefoniert.
		
		