\section{Weitere Anforderungen}

	\subsection{Qualitätsmerkmale}
	Die Applikation soll ohne Einschulung bedienbar sein und die Video- und Audio-Auflösung
	soll abhängig von der Verbindungsqualität skaliert werden.


		\subsubsection{Functionality}
		\begin{description}
			\item[Interoperabilität] Die Applikation soll auf existierende Standards wie 
			SDP\footnote{Javascript Object Notation \cite{IETF-JSON-RFC}}, 
			STUN\footnote{Session Traversal Utilities for NAT \cite{IETF-STUN-RFC}}, HTTP
			und JSON\footnote{JavaScript Object Notation\cite{IETF-JSON-RFC}} setzen und
			wo eigene Standards oder Schnittstellen nötig sind, ist eine Protokoll- oder Schnittstellendokumentation anzufertigen.
			\item[Sicherheit] Die Audio- und Videokommunikation soll verschlüsselt
			werden. Die Verbindung zum Server soll Verschlüsselung unterstützen, sodass
			z.\,B. TLS\footnote{Transport Layer Security, früher SSL} genutzt werden
			könnte.
		\end{description}

		\subsubsection{Usability}
		\begin{description}
			\item[Verständlichkeit] Die Benutzeroberfläche soll so aufgebaut sein, dass
			sich ein Benutzer ohne externe Hilfe innerhalb von Minuten darin
			zurechtfindet. Buttons und Labels sollen sprechend bezeichnet sein.
			\item[Robustheit] Die Oberfläche soll nicht hängen bleiben bei Operationen,
			die länger als 5s dauern, und auf Fehler angemessen reagieren.
		\end{description}

		\subsubsection{Portability}
		\begin{description}
			\item[Anpassbarkeit] Die Applikation soll so gestaltet werden, dass sich die
			Unterstützung für weitere Channel- und Adressbuch-Technologien hinzufügen lässt. Dazu sollen Schnittstellen definiert und dokumentiert werden.
			\item[Installierbarkeit] Die Applikation soll auf jedem
			modernen, WebRTC-fähigen Browser laufen, der sich an die Standards
			hält.
			\item[Austauschbarkeit] Die Oberfläche soll durch eigene Styles einfach umgestaltet und angepasst werden können. Es soll eine Template-Engine verwendet werden, sodass Mehrsprachigkeit später hinzugefügt werden kann.
		\end{description}

	\subsection{Schnittstellen}

		\subsubsection{Signaling-Schnittstelle}
		Die Applikation soll als Signaling-Technologie einen offenen Standard (z.\,B.
		SIP\footnote{Session Initiation Protocol\cite{IETF-SDP-RFC}} oder XAMPP\footnote{Extensible Messaging and Presence Protocol\cite{IETF-XMPP-RFC}} oder JSON über 
		XHR\footnote{XML HTTP Request, asynchrones Nachladen von HTTP-Seiten
		\cite{MDN-XHR-Cross-Site}}) unterstützen.

		\subsubsection{Telefonbuchschnittstelle}
		Die Telefonbuchschnittstelle soll offene und verbreitete Standardformate
		unterstützen (z.\,B. vCard\footnote{Elektronische Visitenkarte
		\cite{IETF-vCard-RFC}}, CSV\footnote{Comma Separated
		Values\cite{IETF-CSV-RFC}}, JSON \ldots).
