\chapter{Projektplan}
Die einzelnen Arbeitspakete werden mithilfe des Projektverwaltungstools Redmine
organisiert. Für die Betreuer wurde ein eigener Zugang zum Redmine-Projekt eingerichtet.

%\section{Gastzugang Redmine}
%http://152.96.192.44/redmine \\
%Zugang: guest / zurich2Rapperswil \\

	\section{Milestones}
		\subsubsection{Elaboration I2 - MS core prototypes}
			\begin{itemize}
				\item Prototypen der Kernkomponenten, insbesondere der Komponenten mit hohen Risiken
					\begin{itemize}
						\item WebRTC P2P Communication Prototype
						\item SIP Server Connection Prototype
						\item Addressbook Interface Implementation Prototype
						\item Channel Interface Implementation Prototype (XHR Simple Queue Server)
					\end{itemize}
				\item Risikoanalyse
				\item Architekturanalyse
			\end{itemize}
			
		\subsubsection{Construction I2 - MS core functionality}
			\begin{itemize}
				\item Implementation der Kernkomponenten (Prototyp-Komponenten-Elaboration)
					\begin{itemize}
						\item P2P Communication
						\item Channel Reference-Implementation
						\item Addressbook Reference-Implementation
					\end{itemize}
				\item Zwischenstand Projektdokumentation
			\end{itemize}
			
		\subsubsection{Construction I4 - MS final app}
			\begin{itemize}
				\item Implementation der Advanced Komponenten
					\begin{itemize}
						\item Media Scaling
						\item Scallable User Interface
					\end{itemize}
				\item Projektdokumentation
			\end{itemize}
		
		\includepdf[pages=-]{../projektplan/media/roadmap.pdf}
		\begin{landscape}
			\includepdf[pages=1]{../projektplan/media/jsvoipcommunication-gantt.pdf}
		\end{landscape}
		\begin{landscape}
			\includepdf[pages=2]{../projektplan/media/jsvoipcommunication-gantt.pdf}
		\end{landscape}
		\begin{landscape}
			\includepdf[pages=3]{../projektplan/media/jsvoipcommunication-gantt.pdf}
		\end{landscape}
		
		
	\section{Aufgewendete Zeit}
		\begin{figure}[H]
			\centering
			\includegraphics[trim=1.75cm 10cm 5.5cm 2.5cm, clip=true,page=1,width=\textwidth]{../projektplan/media/timelog.pdf}
		\end{figure}