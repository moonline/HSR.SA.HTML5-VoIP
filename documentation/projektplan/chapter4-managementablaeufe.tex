\chapter{Managementabläufe}
\section{Kostenvoranschlag}
Das Projekt dauert vom 21. Februar bis 31. Mai 2012 und ist mit 120 Stunden Aufwand pro Teammitglied veranschlagt. Dies macht eine Gesamtarbeitszeit von $$3 \cdot 120h=360h$$ Arbeitsstunden und bei einem Stundensatz von 110 Franken pro Stunde ergibt dies Kosten von $$360h \cdot 110CHF/h=39'600CHF$$ Weitere Ressourcen werden während der Entwicklungszeit nicht benötigt, der Test- und Implementierungsserver wird uns von der Hochschule Rapperswil zur Verfügung gestellt. Auch Kosten wie Anfahrtswege und Ähnliches wurden bewusst weggelassen, da diese im ``normalen`` Studienbetrieb ohnehin anfallen.
Dies ergibt eine durchschnittliche Arbeitszeit von $$\frac{120h}{14W} \simeq 8.5h/W$$ Stunden pro Woche pro Person damit das Projekt rechtzeitig beendet werden kann.

\section{Phasen}
Im Folgenden werden die einzelne Projektabschnitte in einer Übersicht dargestellt, die genauen Einzelschritte und Arbeitspakete der Phasen sind in Redmine erfasst. (Siehe Redmine-Zugang, im Abschnitt \ref{text:redmine} auf Seite \pageref{text:redmine})
\subsection{Inception}
Die Inception beinhaltet in erster Linie eine Idee. Diese wird im Projektantrag formuliert und mit Genehmigung desselben wird die Freigabe für diesen Projektplan erstellt. Während der Inception werden die Arbeitswerkzeuge eingerichtet, die Projektmitglieder als Team organisiert und deren Aufgaben festgelegt. Die Inception wird mit dem Meilenstein 1 (siehe \ref{text:milestone1}) abgeschlossen.
\subsection{Elaboration}
\subsubsection{Iteration 1}
In der ersten Iteration wird der Projektplan gemäss den Ergebnissen der Review überarbeitet. Danach folgt die genaue Ausarbeitung der Use-Cases, des Domainmodells, der Operation Contracts und Ähnlichem.
\subsubsection{Iteration 2}
Während der zweiten Iteration werden die Use-Cases detaillierter (``fully-dressed``) verfasst. Die logische Architektur und das Java-Klassenmodell werden entworfen. Das Ende der Elaboration ist zeitgleich mit dem Meilenstein 3 (siehe \ref{text:milestone3}). 
\subsection{Construction}
\subsubsection{Iteration 1}
Construction ist die Hauptimplementierungsphase. In dieser Zeit werden die während der Elaboration ausgearbeiteten Klassen implementiert und passende Unit-Tests (mithilfe von JUnit 4) ausgearbeitet.
\subsubsection{Iteration 2}
Die zweite Konstruktions-Iteration befasst sich nebst den restlichen Implementierungen mit Usability Tests und Bugfixing.
\subsubsection{Iteration 3}
Die dritte und letzte Iteration befasst sich mit dem Feinschliff. Dazu gehört weiteres Bugfixing und Systemtests. Gegen Ende dieser Iteration erfolgt ein ``Code-Freeze`` in dem kein weiteres Feature mehr hinzugefügt werden darf und nur noch Verbesserungen auf Bug-Ebene erlaubt sind. Siehe dazu auch Meilenstein 4 (Abschnitt \ref{text:milestone4})
\subsection{Transition}
Die Abschlussphase. In dieser geht es Hauptsächlich um die Erstellung der Benutzerdokumentation und die Vorbereitungen für die Schlusspräsentation. Sie wird dann auch vom Meilenstein 5 (siehe \ref{text:finish}) abgeschlossen.
\section{Meilensteine}
\subsection{Review Projektplan}
\label{text:milestone1}
Termin: 8. März 2012 13:10 Uhr \\
Review Projektplan mit Zeitplan und aktuellen Iterationsplänen\\
Der Projektplan wurde zu dem Zeitpunkt bereits abgegeben und an diesem Termin mit dem Betreuer besprochen. Nach erfolgtem Review verläuft das weitere Projekt gemäss Projektplan.

\subsection{Anforderungen und Analyse}
Termin: 20. März 2012 14:05 Uhr \\
Review der Anforderungsspezifikation und der Domainanalyse\\
Zu diesem Zeitpunkt stehen das Domainmodell (als konzeptionelles Klassendiagramm) und sämtliche Use-Cases (im ``brief``-Format, die wichtigen ``fully-dressed``) fest. Die Soll-Szenarien sind bekannt und als ``Features`` im Redmine erfasst.
\subsection{Ende Elaboration}
\label{text:milestone3}
Termin: 3. April 2012 14:05 Uhr \\
Zwischenpräsentation mit Demo eines Architekturprototypen, Review\\

\subsection{Erster lauffähiger Prototyp}
Termin: 3. April 2012 \\
Zu diesem Zeitpunkt existiert eine rudimentäre Version der Software die die Core-Funktionalität besitzt, soll heissen min ein lauffähiger Test, Fetchen der Website funktioniert und es wird ein Report (z.B. auf stdout) angezeigt
Dieser Meilenstein fällt mit dem Milestone ``Ende Elaboration`` zusammen.

\subsection{Architektur \& Design}
\label{text:architecture}
Termin: 24. April 2012 14:05 Uhr \\
Review von Architektur/Design und Architekturdokumentation\\

\subsection{Codefreeze}
\label{text:milestone4}
Termin: 18. April 2012 23:59 Uhr \\
Software soll soweit ``fertig`` sein. Es werden keine neuen Features mehr implementiert, nur noch Bugfixing betrieben. Zehn Tage Puffer soll zum Vollenden der Doku und Vorbereiten der Schlusspräsentation genutzt werden.

\subsection{Schlusspräsentation}
\label{text:finish}
Termin: 29. Mai 13:00 Uhr \\
Präsentation und Demo der Software\\

\section{Besprechungen}
Das Team trifft sich regelmässig jeweils am Dienstagnachmittag von 13:10 bis 16:50 Uhr im dem im Übungsraum \room, um sowohl den aktuellen Stand der Dinge zu besprechen als auch gemeinsame Tätigkeiten wie das Domain- und Klassenmodell auszuarbeiten durchzuführen. Die eventuell übrig bleibende Zeit wird von den Teammitgliedern individuell genutzt, um das Projekt in den ihnen zugewiesenen Aufgabenbereichen voranzutreiben.
